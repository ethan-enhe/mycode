\documentclass[UTF8]{ctexart}
\usepackage{algorithm}
\usepackage{algorithmic}
\usepackage{cancel} 
\usepackage{geometry}
\usepackage{graphicx}
\usepackage{listings}
\usepackage[dvipsnames]{xcolor}
% 在导言区进行样式设置
\lstset{
        language=C++, % 设置语言
        basicstyle=\ttfamily, % 设置字体族
        breaklines=true, % 自动换行
        keywordstyle=\bfseries\color{NavyBlue}, % 设置关键字为粗体,颜色为 NavyBlue
        morekeywords={}, % 设置更多的关键字,用逗号分隔
        emph={self}, % 指定强调词,如果有多个,用逗号隔开
        emphstyle=\bfseries\color{Rhodamine}, % 强调词样式设置
        commentstyle=\itshape\color{black!50!white}, % 设置注释样式,斜体,浅灰色
        stringstyle=\bfseries\color{PineGreen!90!black}, % 设置字符串样式
        columns=flexible,
        numbers=left, % 显示行号在左边
        numbersep=2em, % 设置行号的具体位置
        numberstyle=\footnotesize, % 缩小行号
        frame=single, % 边框
        framesep=1em % 设置代码与边框的距离
}
\title{}
\author{周翟恩和 PB22000008}
\date{\today}
\begin{document}
\maketitle
\section{Balanced array}

\subsection{legend}
A \textit{balanced array} is defined as an integer array $ a_1,a_2,\ldots a_l $ that satisfies the following condition:
\begin{itemize}
        \item There exists an integer $k$, such that $ 1\le k\le \frac {l-1} 2$
        \item $ a_i+a_{i+2k}=2a_{i+k} $ for each $ i $ in $ 1,2,\ldots l-2k $ 
\end{itemize}

Given an array $ A_1,A_2,\ldots A_n $, the task is to determine whether $ A_{1\ldots i} $ is a balanced array for each $ i $ in $ 1,2,\ldots n $.

To minimize the size of the input file,\textbf{ \( A_i \) was encoded in base-62}, where the characters \( 0\ldots 9 \textit{a}\ldots \textit{z} \textit{A}\ldots \textit{Z} \) correspond to the numerical values \( 1\ldots 62 \) for each digit.

\subsection{input}

The first line contains an integer $ n $ ($ 1\le n\le 10^7 $), denoting the length of array $ A $.

The second line contains $ n $ integers $ A_1,A_{2}\ldots A_{n} $ ($ 1\le A_i\le 10^9 $).

\subsection{output}

Output a 0/1 string $ s_{1\ldots n} $, such that $ s_i=\texttt{'1'} $ if $ A_{1\ldots i} $ is balanced, $ s_i=\texttt{'0'} $ otherwise.



\end{document}
